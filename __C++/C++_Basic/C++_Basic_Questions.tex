\documentclass[UTF8,a4paper,8pt]{ctexart} 

 \usepackage{graphicx}%学习插入图
 \usepackage{verbatim}%学习注释多行
 \usepackage{booktabs}%表格
 \usepackage{geometry}%图片
 \usepackage{amsmath} 
 \usepackage{amssymb}
 \usepackage{listings}%代码
 \usepackage{xcolor}  %颜色
 \usepackage{enumitem}%列表格式
 \CTEXsetup[format+={\flushleft}]{section}


\geometry{left=1.6cm,right=1.8cm,top=2cm,bottom=1.7cm} %设置文章宽度

\pagestyle{plain} 		  %设置页面布局
\author{郑华}
\title{C++  Questions }
 %代码效果定义
 \definecolor{codegreen}{rgb}{0,0.6,0}
 \definecolor{codegray}{rgb}{0.5,0.5,0.5}
 \definecolor{codepurple}{rgb}{0.58,0,0.82}
 \definecolor{backcolour}{rgb}{0.95,0.95,0.92}
 
 \lstdefinestyle{mystyle}{
 	backgroundcolor=\color{backcolour},   
 	commentstyle=\color{codegreen},
 	keywordstyle=\color{magenta},
 	numberstyle=\tiny\color{codegray},
 	stringstyle=\color{codepurple},
 	basicstyle=\footnotesize,
 	breakatwhitespace=false,         
 	breaklines=true,                 
 	captionpos=b,                    
 	keepspaces=true,                 
 	%numbers=left,                    
 	%numbersep=5pt,                  
 	showspaces=false,                
 	showstringspaces=false,
 	showtabs=false,                  
 	tabsize=2
 }
\lstset{style=mystyle, escapeinside=``}

\begin{document}          %正文排版开始
 	\maketitle
	
\section*{面向对象的设计思想}
	 \paragraph{1.面向对象包含哪些基本概念?}
	 \paragraph{2.面向对象的基本特征包括哪些?}
	 \paragraph{3.为什么封装对面向对象来说很重要?}
	 \paragraph{4.接口与实现的分离有什么好处?}
	 \paragraph{5.抽象是什么?}
	 \paragraph{6.封装与抽象有什么联系?}
	 \paragraph{7.继承是否会消弱封装机制?}
	 \paragraph{8.组合是怎么定义的,它有什么作用,它包括什么?}
	 \paragraph{9.如何理解动态特性?}
	 
\section*{C++中的类和对象}
	 \paragraph{1.如何理解对象的初始化?}
	 \paragraph{2.如何理解对象的生存周期?}
	 \paragraph{3.如何向函数传递对象?}
	 \paragraph{4.编写C++ 类时需要注意哪些问题?}
	 \paragraph{5.如何理解构造函数?}
	 \paragraph{6.默认构造函数是什么,它有什么特点?}
	 \paragraph{7.何时调用拷贝构造函数?}
	 \paragraph{8.深拷贝和浅拷贝的区别是什么?}
	 \paragraph{9.如何理解析构函数?}
	 \paragraph{10.程序如何处理静态成员变量及静态成员函数?}
	 \paragraph{11.为什么使用友元,如何使用?}
	 \paragraph{12.使用友元的优点}  
	 
\section*{继承}
	 \paragraph{1.C++ 的继承结构是怎样的?}
	 \paragraph{2.当派生类与基类成员名称冲突时应如何解决?}
	 \paragraph{3.为什么要使用虚基类?}
	 \paragraph{4.继承体系中构造函数的调用顺序是怎样的?}
	 \paragraph{5.类的默认访问权限是什么?为什么使用它作为默认权限?}
	 \paragraph{6.为什么要使用Protected,如何使用?}
	 \paragraph{7.为什么派生类不能访问基类的private成员?}
	 \paragraph{8.struct 与 class 有什么不同?}
	 \paragraph{9.如何为基类构造函数传递参数?}    
	 
\section*{多态}
	 \paragraph{1.如何理解C++中的捆绑?} 
	 \paragraph{2.如何理解和使用虚函数?}
	 \paragraph{3.如何理解和使用纯虚函数?}
	 \paragraph{4.如何理解和使用抽象类?}
	 \paragraph{5.多态是如何实现的?}
	 \paragraph{6.如何理解静态多态和动态多态?}
	 \paragraph{7.如何理解虚函数和构造函数?}
	 \paragraph{8.如何理解虚函数和析构函数?}

\section*{指针和字符串}
	 \paragraph{1.如何理解sizeof关键字?} 
	 \paragraph{2.指针是什么?}
	 \paragraph{3.如何理解地址和指针的关系?}
	 \paragraph{4.指针和取值操作符\&如何结合使用?}
	 \paragraph{5.指针的运算有哪些?}
	 \paragraph{6.指针变量和引用有什么区别?}
	 \paragraph{7.指针变量与变量指针有什么区别?}
	 \paragraph{8.指针的比较指的是什么?}
	 \paragraph{9.如何使用函数指针?}  
	 \paragraph{10.如何理解指针函数?}
	 \paragraph{11.如何理解指针数组与数组指针?}
	 \paragraph{12.使用指针有哪些常见的错误?}
	 \paragraph{13.常用的字符串操作函数有哪些?}
	 \paragraph{14.如何理解字符数组和字符指针?}
 
\section*{运算符重载}
 	 \paragraph{1.为什么使用运算符重载?} 
 	 \paragraph{2.使用运算符重载应遵循哪些规则?}
 	 \paragraph{3.哪些运算符可以重载,哪些可以重载?}
 	 \paragraph{4.为什么要使用友元函数重载运算符?}
 	 \paragraph{5.使用友元函数重载“++”“--”运算符可能会出现什么问题?}
 	 \paragraph{6.如何实现new 和 delete 运算符的重载?}
 	 \paragraph{7.如何重载数组下标运算符?}
 	 \paragraph{8.如何将运算符函数作为成员函数使用?}
 	 \paragraph{9.成员运算符函数与友元运算符函数有什么区别?} 
 	 
\section*{用户自定义的数据类型和枚举}
  	 \paragraph{1.什么事枚举,如何使用枚举?} 

\section*{类型转换和RTTl}
 	 \paragraph{1.C++预定义的类型转换有哪些方式?} 
 	 \paragraph{2.如何实现类这种数据类型与其他数据类型的装换?}
 	 \paragraph{3.为什么需要转换函数,如何创建转换函数?}
 	 \paragraph{4.c++定义了哪几个强制转换操作符?作用分别是什么?}
 	 \paragraph{5.如何区分静态类型检查和动态类型检查?}
 	 \paragraph{6.为什么要避免使用动态类型检查?}
 	 \paragraph{7.什么事运行时类型标识?}
 	 \paragraph{8.为什么向下的类型转换是危险的?}
 	 \paragraph{9.dynamic\_cast$\langle T \rangle $()函数的作用是什么?}
 	 \paragraph{10.static\_cast$\langle T \rangle $()函数的作用是什么?}
 	 \paragraph{11.typeid()的作用是什么?} 

\section*{异常处理}
	 \paragraph{1.C++ 异常处理的原理是什么?} 
	 \paragraph{2.异常处理是如何实现的?}
	 \paragraph{3.使用异常时应该注意哪些方面?}
	 \paragraph{4.抛出的异常和捕获的异常是否必须匹配?}
	 \paragraph{5.如何处理Try语句中函数抛出的异常?}
	 \paragraph{6.程序在何时执行catch语句?}
	 \paragraph{7.一个try语句是否可以使用多个catch语句?如何使用?}
	 \paragraph{8.对异常使用省略号有什么作用?}
	 \paragraph{9.throw 语句具有什么作用?}
	 \paragraph{10.如何实现重新抛出异常?}
	 \paragraph{11.构造和析构对象时产生的异常应该如何处理?}
	 \paragraph{12.如何使用默认函数参数避免异常和错误发生?}
	 \paragraph{13.处理异常时terminate() 函数和 unexpected()函数分别有什么作用?}

\section*{标准模版库}
	 \paragraph{1.什么是标准模版库?为什么要使用标准模板库?} 
	 \paragraph{2.标准模版库包含哪些头文件?}
	 \paragraph{3.如何理解容器?}
	 \paragraph{4.标准模版库中容器的存储方式和访问方式}
	 \paragraph{5.标准模版库中的容器是如何实现的?}
	 \paragraph{6.关联容器是如何工作的?}
	 \paragraph{7.迭代器在标准模版库设计中有什么作用?}
	 \paragraph{8.如何理解输入输出迭代器?}
	 \paragraph{9.STL 包括哪些算法?}
	 \paragraph{10.vector,list,deque,set,map的内部实现和使用}

\section*{通用函数及模版}
  	 \paragraph{1.如何理解模版?} 
  	 \paragraph{2.如何显式重载通用函数?}
  	 \paragraph{3.什么情况下不能使用通用函数代替重载函数?}
  	 \paragraph{4.多个文件之间是否可以编译相同的函数模版定义?}
  	 \paragraph{5.类模板和模板类之间有什么关系?}
  	 \paragraph{6.当函数模版与同名非函数模版函数重载时如何进行调用?}
  	 \paragraph{7.如何使用模版定义通用类?}
  	 \paragraph{8.是否可以创建含有多个通用数据类型的通用类?}
  	 \paragraph{9.创建含有参数的操作符需要注意什么?}
  	 \paragraph{10.在函数模版中如何使用数组作为参数?}

\section*{输入和输出}
  	 \paragraph{1.} 
  	 \paragraph{2.}
  	 \paragraph{3.}
  	 \paragraph{4.}
  	 \paragraph{5.}
  	 \paragraph{6.}
  	 \paragraph{7.}
  	 \paragraph{8.}

\section*{内存管理}
  	 \paragraph{1.} 
  	 \paragraph{2.}
  	 \paragraph{3.}
  	 \paragraph{4.}
  	 \paragraph{5.}
  	 \paragraph{6.}
  	 \paragraph{7.}
  	 \paragraph{8.}
 
\section*{进程与线程}
  	 \paragraph{1.} 
  	 \paragraph{2.}
  	 \paragraph{3.}
  	 \paragraph{4.}
  	 \paragraph{5.}
  	 \paragraph{6.}
  	 \paragraph{7.}
  	 \paragraph{8.}

\section*{关于性能的思考}       
  	 \paragraph{1.} 
  	 \paragraph{2.}
  	 \paragraph{3.}
  	 \paragraph{4.}
  	 \paragraph{5.}
  	 \paragraph{6.}
  	 \paragraph{7.}
  	 \paragraph{8.}
  	 
\end{document} 
 		    